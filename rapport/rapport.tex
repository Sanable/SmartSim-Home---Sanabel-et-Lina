% Options for packages loaded elsewhere
\PassOptionsToPackage{unicode}{hyperref}
\PassOptionsToPackage{hyphens}{url}
%
\documentclass[
]{article}
\usepackage{amsmath,amssymb}
\usepackage{lmodern}
\usepackage{iftex}
\ifPDFTeX
  \usepackage[T1]{fontenc}
  \usepackage[utf8]{inputenc}
  \usepackage{textcomp} % provide euro and other symbols
\else % if luatex or xetex
  \usepackage{unicode-math}
  \defaultfontfeatures{Scale=MatchLowercase}
  \defaultfontfeatures[\rmfamily]{Ligatures=TeX,Scale=1}
\fi
% Use upquote if available, for straight quotes in verbatim environments
\IfFileExists{upquote.sty}{\usepackage{upquote}}{}
\IfFileExists{microtype.sty}{% use microtype if available
  \usepackage[]{microtype}
  \UseMicrotypeSet[protrusion]{basicmath} % disable protrusion for tt fonts
}{}
\makeatletter
\@ifundefined{KOMAClassName}{% if non-KOMA class
  \IfFileExists{parskip.sty}{%
    \usepackage{parskip}
  }{% else
    \setlength{\parindent}{0pt}
    \setlength{\parskip}{6pt plus 2pt minus 1pt}}
}{% if KOMA class
  \KOMAoptions{parskip=half}}
\makeatother
\usepackage{xcolor}
\usepackage{longtable,booktabs,array}
\usepackage{multirow}
\usepackage{calc} % for calculating minipage widths
% Correct order of tables after \paragraph or \subparagraph
\usepackage{etoolbox}
\makeatletter
\patchcmd\longtable{\par}{\if@noskipsec\mbox{}\fi\par}{}{}
\makeatother
% Allow footnotes in longtable head/foot
\IfFileExists{footnotehyper.sty}{\usepackage{footnotehyper}}{\usepackage{footnote}}
\makesavenoteenv{longtable}
\usepackage{graphicx}
\makeatletter
\def\maxwidth{\ifdim\Gin@nat@width>\linewidth\linewidth\else\Gin@nat@width\fi}
\def\maxheight{\ifdim\Gin@nat@height>\textheight\textheight\else\Gin@nat@height\fi}
\makeatother
% Scale images if necessary, so that they will not overflow the page
% margins by default, and it is still possible to overwrite the defaults
% using explicit options in \includegraphics[width, height, ...]{}
\setkeys{Gin}{width=\maxwidth,height=\maxheight,keepaspectratio}
% Set default figure placement to htbp
\makeatletter
\def\fps@figure{htbp}
\makeatother
\usepackage[normalem]{ulem}
\setlength{\emergencystretch}{3em} % prevent overfull lines
\providecommand{\tightlist}{%
  \setlength{\itemsep}{0pt}\setlength{\parskip}{0pt}}
\setcounter{secnumdepth}{-\maxdimen} % remove section numbering
\ifLuaTeX
  \usepackage{selnolig}  % disable illegal ligatures
\fi
\IfFileExists{bookmark.sty}{\usepackage{bookmark}}{\usepackage{hyperref}}
\IfFileExists{xurl.sty}{\usepackage{xurl}}{} % add URL line breaks if available
\urlstyle{same} % disable monospaced font for URLs
\hypersetup{
  hidelinks,
  pdfcreator={LaTeX via pandoc}}

\author{}
\date{}

\begin{document}

\begin{longtable}[]{@{}
  >{\raggedright\arraybackslash}p{(\columnwidth - 2\tabcolsep) * \real{0.5000}}
  >{\raggedright\arraybackslash}p{(\columnwidth - 2\tabcolsep) * \real{0.5000}}@{}}
\toprule()
\begin{minipage}[b]{\linewidth}\raggedright
\includegraphics[width=0.63611in,height=5.26111in]{vertopal_e39372513fc74674888784101d54e19c/media/image1.png}
\end{minipage} & \begin{minipage}[b]{\linewidth}\raggedright
\begin{quote}
\includegraphics[width=4.24028in,height=0.90694in]{vertopal_e39372513fc74674888784101d54e19c/media/image2.png}
\end{quote}

Licence MIASHS deuxième année

\begin{quote}
\textbf{Rapport de projet informatique Smart Sim Home :}\\
\textbf{Le Simulateur de maison}\\
\textbf{intelligente}

\textbf{Lien GitHub :}
\end{quote}\strut
\end{minipage} \\
\midrule()
\endhead
\bottomrule()
\end{longtable}

\textbf{Membres du groupe}

\begin{quote}
\textbf{ABDELQADER}\\
\textbf{Sanabel 44011246}\\
\textbf{Lina} \textbf{BOULIFA}\\
\textbf{44012836}
\end{quote}

\begin{longtable}[]{@{}
  >{\raggedright\arraybackslash}p{(\columnwidth - 2\tabcolsep) * \real{0.5000}}
  >{\raggedright\arraybackslash}p{(\columnwidth - 2\tabcolsep) * \real{0.5000}}@{}}
\toprule()
\begin{minipage}[b]{\linewidth}\raggedright
\begin{quote}
\includegraphics[width=2.19722in,height=2.92639in]{vertopal_e39372513fc74674888784101d54e19c/media/image3.png}
\end{quote}
\end{minipage} & \begin{minipage}[b]{\linewidth}\raggedright
\includegraphics[width=2.14583in,height=2.86389in]{vertopal_e39372513fc74674888784101d54e19c/media/image4.png}
\end{minipage} \\
\midrule()
\endhead
\bottomrule()
\end{longtable}

\begin{quote}
\includegraphics[width=2.22917in,height=0.28194in]{vertopal_e39372513fc74674888784101d54e19c/media/image5.png}
\end{quote}

\begin{longtable}[]{@{}
  >{\raggedright\arraybackslash}p{(\columnwidth - 2\tabcolsep) * \real{0.5000}}
  >{\raggedright\arraybackslash}p{(\columnwidth - 2\tabcolsep) * \real{0.5000}}@{}}
\toprule()
\begin{minipage}[b]{\linewidth}\raggedright
\begin{quote}
\textbf{1\hspace{0pt} Introduction}\hspace{0pt}\\
\textbf{2\hspace{0pt} Environnement de travail}\hspace{0pt}\\
\textbf{3\hspace{0pt} Description du projet et objectifs}\hspace{0pt}\\
\textbf{4\hspace{0pt} Bibliothèques, Outils et
technologies}\hspace{0pt}\\
\textbf{5\hspace{0pt} Travail réalisé}\hspace{0pt}\\
\textbf{6\hspace{0pt} Difficultés rencontrées}\hspace{0pt}\\
\textbf{7\hspace{0pt} Bilan}\hspace{0pt}\\
7.1\hspace{0pt} Conclusion . . . . . . . . . . . . . . . . . . . . . . .
. . . . . . . . . . .\hspace{0pt} 7.2\hspace{0pt} Perspectives . . . . .
. . . . . . . . . . . . . . . . . . . . . . . . . . .
.\hspace{0pt}\textbf{8\hspace{0pt} Bibliographie}\hspace{0pt}\\
\textbf{9\hspace{0pt} Webographie}\hspace{0pt}\\
\textbf{10\hspace{0pt}Annexes}\hspace{0pt}\\
\textbf{A Cahier des charges}\hspace{0pt}\\
\textbf{B\hspace{0pt} Exemple d\textquotesingle exécution du
projet}\hspace{0pt}\\
\textbf{C\hspace{0pt} Manuel utilisateur}\hspace{0pt}
\end{quote}\strut
\end{minipage} & \begin{minipage}[b]{\linewidth}\raggedright
\begin{quote}
\textbf{4}\\
\textbf{4}\\
4\\
\textbf{4}\\
\textbf{4}\\
\textbf{4}\\
\textbf{4}\\
4\\
4\\
\textbf{5}\\
\textbf{6}\\
\textbf{7}\\
\textbf{7}\\
\textbf{7}\\
\textbf{7}
\end{quote}\strut
\end{minipage} \\
\midrule()
\endhead
\bottomrule()
\end{longtable}

\begin{quote}
\includegraphics[width=0.17639in,height=0.125in]{vertopal_e39372513fc74674888784101d54e19c/media/image6.png}\includegraphics[width=1.52083in,height=0.26042in]{vertopal_e39372513fc74674888784101d54e19c/media/image8.png}

\emph{\textbf{\uline{INTRODUCTION}}}

Dans un monde où la technologie cherche à se fondre avec notre
environnement, l'assistant vocale est devenu une interface familière
dans la domotique. Notre projet SmartSim Home s'inscrit dans cette
tendance en allant un peu plus loin : là où des systèmes comme Alexa se
contentent de répondre à des commandes, notre simulateur cherche à
s'adapter au contexte et à anticiper les besoins de l'utilisateur pour
améliorer son quotidien.

Concrètement, SmartSim Home simule le fonctionnement d'une maison
connectée capable de contrôler différents éléments du quotidien :
température, lumière, volets et équipements électriques, le tout
accessible via une interface web. Le simulateur ne se limite pas au
confort domestique : il vise également à optimiser la consommation
d'énergie à l'aide de modes spécifiques comme le mode économique, tout
en contribuant à l'amélioration du cadre de vie des occupants. Il
intègre plusieurs capteurs simulés, notamment des capteurs de
température, de luminosité, d'humidité, mais aussi un détecteur de
monoxyde de carbone, un gaz dangereux et potentiellement mortel.

Ce projet a été développé à l'aide d'outils d'IA générative et reste
théorique : il s'agit d'un simulateur et non d'une installation réelle.

Toutefois, il a été conçu dans l'optique d'évoluer. À terme, l'objectif
serait de permettre au système de communiquer avec l'utilisateur à la
manière d'un assistant vocal de type \emph{Alexa}, afin de le prévenir,
lui donner des conseils ou signaler une anomalie en temps réel.
\end{quote}

\begin{longtable}[]{@{}
  >{\raggedright\arraybackslash}p{(\columnwidth - 4\tabcolsep) * \real{0.3333}}
  >{\raggedright\arraybackslash}p{(\columnwidth - 4\tabcolsep) * \real{0.3333}}
  >{\raggedright\arraybackslash}p{(\columnwidth - 4\tabcolsep) * \real{0.3333}}@{}}
\toprule()
\begin{minipage}[b]{\linewidth}\raggedright
\textbf{2\hspace{0pt}}
\end{minipage} & \begin{minipage}[b]{\linewidth}\raggedright
\begin{quote}
\textbf{Environnement de travail}
\end{quote}
\end{minipage} &
\multirow{2}{*}{\begin{minipage}[b]{\linewidth}\raggedright
\includegraphics[width=2.26111in,height=1.40555in]{vertopal_e39372513fc74674888784101d54e19c/media/image7.png}
\end{minipage}} \\
\multicolumn{2}{@{}>{\raggedright\arraybackslash}p{(\columnwidth - 4\tabcolsep) * \real{0.6667} + 2\tabcolsep}}{%
\begin{minipage}[b]{\linewidth}\raggedright
\begin{quote}
La collaboration au sein du binôme s'est organisée principalement sur
Google Drive. Nous avons organisé notre espace de travail avec une
arborescence de dossiers dédiés (documentation, codes,
algorithmes\ldots) permettant d'accéder aux ressources de manière
immédiate. Cette application a servi de cahier de bord central pour le
projet : nous y mettions toutes les idées fondamentales, les extraits de
code clés, et le suivi détaillé des étapes à réaliser. Chaque avancée
par un membre du binôme était documentée et partagée dans ce dossier
commun,
\end{quote}
\end{minipage}} \\
\midrule()
\endhead
\bottomrule()
\end{longtable}

\begin{quote}
permettant une vision synchronisée de l'avancement pour les deux membres
de l'équipe.

Pour la gestion de notre code source, nous avons également utilisé Git
comme système de versionnage, afin de travailler de manière ordonnée sur
les mêmes fichiers, de suivre l\textquotesingle historique de nos
modifications et de faciliter la collaboration.

Dès le lancement du projet, nous nous sommes rencontrées à plusieurs
reprises pour des séances de brainstorming. Ces réunions en face-à-face
nous ont permis de préciser l\textquotesingle idée générale, de définir
les principales fonctionnalités et d\textquotesingle établir une
première feuille de route.

Concernant l'environnement de développement, le projet a été
principalement codé sur une machine macOS, équipée d'une puce Apple M3,
du côté de Sanabel. Le développement a été réalisé à l'aide de l'éditeur
Visual Studio Code, choisi pour sa polyvalence, sa légèreté et son large
écosystème d'extensions facilitant le travail en Python.

\includegraphics[width=0.17639in,height=0.125in]{vertopal_e39372513fc74674888784101d54e19c/media/image6.png}\includegraphics[width=1.52083in,height=0.26042in]{vertopal_e39372513fc74674888784101d54e19c/media/image8.png}

De son côté, Lina a recréé l'intégralité de l'arborescence du projet sur
son ordinateur Windows, afin de pouvoir effectuer des modifications et
des ajustements. Cette duplication était nécessaire, le projet ayant
évolué progressivement au fil du temps, avec des ajouts et des
réorganisations fréquentes du code et des fichiers.

Le projet a été codé en Python 3, et a nécessité pour son exécution la
présence de l'interpréteur Python dans sa version 3.11 ou supérieure
comme pré requis avant l'installation des\\
dépendances. Sanabel a utilisé la version 3.12.7.

La documentation technique et le rapport ont été rédigésen LaTeX via la
plateforme Overleaf. Ce choix a été motivé par la qualité de mise en
page requise pour un rapport formel et la gestion automatisée de la
bibliographie.

\textbf{3\hspace{0pt}Description du projet et objectifs}

SmartSim Home est un simulateur de maison intelligente. Son objectif
principal est de reproduire le fonctionnement d'un écosystème domotique
dans un environnement virtuel et contrôlé. Il se distingue des autres
systèmes similaires de par sa capacité à anticiper les besoins de
l'utilisateur plutôt que de se limiter à y répondre. Le projet
s'articule autour de quatre piliers fondamentaux :

●\hspace{0pt} Centraliser la surveillance de plusieurs capteurs
virtuels, qui permettent de mesurer la luminosité, la présence d'un
individu dans une pièce, l'humidité, la qualité de l'air ou encore le
monoxyde de carbone.

●\hspace{0pt} Assurer la commande de plusieurs actionneurs simulés
incluant les systèmes d'éclairage, de chauffage et de climatisation, les
volets, prises électriques et un système d'alarme.

●\hspace{0pt} Activer des modes de fonctionnement prédéfinis, comme un
mode automatique qui permet de tout gérer, un mode éco pour réduire la
consommation, un mode vacances et une alarme qui assure la sécurité des
habitants. En effet, elle retentit lorsqu'un capteur détecte un niveau
de monoxyde de carbone anormalement élevé.

●\hspace{0pt} Intégrer une intelligence artificielle locale via Ollama,
avec à terme la possibilité d\textquotesingle interagir par commande
vocale ou textuelle.

L'ensemble de ces mécanismes est visualisé via une interface utilisateur
unifiée, qui sert de tableau de bord à ce projet.

\textbf{4\hspace{0pt}Bibliothèques, Outils et technologies}

Le projet repose sur une architecture qui combine un serveur web
embarqué et des bibliothèques Python spécialisées pour créer un
environnement de simulation interactif comme le nôtre.

L'application intègre un serveur web utilisant le modèle Python standard
http.server avec les bases HTTPServer et BaseHTTPRequestHandler. Ce
serveur délivre\\
localement une interface utilisateur développée en HTML, CSS et
JavaScript, accessible à l'adresse .

Le projet repose sur plusieurs bibliothèques Python :

\includegraphics[width=0.17639in,height=0.125in]{vertopal_e39372513fc74674888784101d54e19c/media/image6.png}\includegraphics[width=1.52083in,height=0.26042in]{vertopal_e39372513fc74674888784101d54e19c/media/image8.png}

○\hspace{0pt} threading et time pour gérer les processus simultanés et
la synchronisation ○\hspace{0pt} random pour générer des valeurs
réalistes des capteurs virtuels\\
○\hspace{0pt} json pour structurer et échanger les données entre le
serveur et l'interface ○\hspace{0pt} pyttsx3 pour implémenter des
réponses par synthèse vocale\\
○\hspace{0pt} ollama pour une assistance par IA locale, permettant
d'enrichir les interactions
\end{quote}

\begin{longtable}[]{@{}
  >{\raggedright\arraybackslash}p{(\columnwidth - 0\tabcolsep) * \real{1.0000}}@{}}
\toprule()
\begin{minipage}[b]{\linewidth}\raggedright
\begin{quote}
\includegraphics[width=6.59444in,height=2.16667in]{vertopal_e39372513fc74674888784101d54e19c/media/image9.png}○\hspace{0pt}
\end{quote}
\end{minipage} \\
\midrule()
\endhead
\bottomrule()
\end{longtable}

\begin{quote}
Cette combinaison d'outils permet d\textquotesingle exécuter l'ensemble
du simulateur directement en local sur une seule machine.
\end{quote}

\begin{longtable}[]{@{}
  >{\raggedright\arraybackslash}p{(\columnwidth - 4\tabcolsep) * \real{0.3333}}
  >{\raggedright\arraybackslash}p{(\columnwidth - 4\tabcolsep) * \real{0.3333}}
  >{\raggedright\arraybackslash}p{(\columnwidth - 4\tabcolsep) * \real{0.3333}}@{}}
\toprule()
\begin{minipage}[b]{\linewidth}\raggedright
\textbf{5\hspace{0pt}}
\end{minipage} &
\multicolumn{2}{>{\raggedright\arraybackslash}p{(\columnwidth - 4\tabcolsep) * \real{0.6667} + 2\tabcolsep}@{}}{%
\begin{minipage}[b]{\linewidth}\raggedright
\begin{quote}
\textbf{Travail réalisé}
\end{quote}
\end{minipage}} \\
\midrule()
\endhead
\multicolumn{2}{@{}>{\raggedright\arraybackslash}p{(\columnwidth - 4\tabcolsep) * \real{0.6667} + 2\tabcolsep}}{%
\includegraphics[width=1.63611in,height=3.13611in]{vertopal_e39372513fc74674888784101d54e19c/media/image10.png}}
& \begin{minipage}[t]{\linewidth}\raggedright
\begin{quote}
L'objectif principal du travail était de concevoir une architecture
logicielle fonctionnelle et évolutive, capable de gérer des capteurs,
des actionneurs et différents modes de fonctionnement, tout en préparant
la future intégration de l'intelligence artificielle et de l'interaction
vocale.

Dans un premier temps, nous avons sollicité ChatGPT et DeepSeek afin de
nous aider à générer et structurer le code initial du projet. Ces outils
nous ont\\
principalement guidés dans la mise en place de l'arborescence du projet,
la création des différents dossiers, des fichiers Python ainsi que dans
le choix et l'organisation des bibliothèques nécessaires. Cette phase a
permis d'obtenir une base de travail solide et structurée.

Le développement reposait sur un processus itératif de copier-coller du
code suivi de tests et d'adaptations manuelles. Au fil du projet, le
code a été affiné et enrichi progressivement avec des fonctionnalités
supplémentaires. Les modes spécifiques tels que le mode automatique, le
mode éco ou le mode vacances ont ainsi été implémentés, permettant
d'adapter le comportement de la maison en fonction du
\end{quote}\strut
\end{minipage} \\
\bottomrule()
\end{longtable}

\begin{quote}
contexte.

Nous avons également commencé à travailler sur l'intégration d'une
intelligence artificielle locale via Ollama. Son objectif à terme est de
permettre une interaction plus naturelle entre l'utilisateur et la
maison intelligente. À ce stade du projet, l'IA est intégrée au code
mais son utilisation reste limitée et expérimentale.OLLAMA répond à
certaine question poser par l'utilisateur concernant la maison
malheureusement sa réactivité est encore limitée.

Concernant l'interaction avec l'utilisateur, la commande vocale n'est
pas encore opérationnelle. Il n'est actuellement pas possible de
dialoguer directement avec l'application ni de formuler des

\includegraphics[width=0.17639in,height=0.125in]{vertopal_e39372513fc74674888784101d54e19c/media/image6.png}\includegraphics[width=1.52083in,height=0.26042in]{vertopal_e39372513fc74674888784101d54e19c/media/image8.png}

requêtes vocales ou écrites à destination de l'assistant. L'interaction
se fait donc principalement à travers l'interface web, à l'aide de
boutons qui permettent d'activer ou de désactiver certains modes et
équipements. Celle-ci permet de visualiser l'état des capteurs, des
actionneurs et des alertes, rendant la simulation exploitable et
compréhensible.

Le but final du projet est de parvenir à une maison intelligente
totalement autonome et pilotable à la voix, capable de comprendre les
demandes de l'utilisateur, de dialoguer de manière naturelle et de
prendre des décisions sans intervention manuelle. Le travail réalisé
constitue donc une base solide, servant de fondation pour des
améliorations futures.

\textbf{Détail des modes implémentés}

Le mode Éco a été conçu dans le but de réduire la consommation
énergétique de la maison intelligente tout en maintenant un niveau de
confort acceptable. Ce mode peut être activé manuellement par
l'utilisateur à l'aide d'un bouton situé dans la section ``Modes'' de
l'interface web.

Une fois le mode Éco activé, plusieurs règles spécifiques sont
appliquées automatiquement au système :

●\hspace{0pt} Le chauffage ne s'active que lorsque la température
descend en dessous de 18 °C, ce qui permet d'éviter une utilisation
excessive du chauffage.\hspace{0pt}

●\hspace{0pt} La climatisation ne s'active qu'à partir de 27 °C, ce qui
oblige à l\textquotesingle utiliser lorsqu'il est réellement
nécessaire.\hspace{0pt}

●\hspace{0pt} L'intensité lumineuse est volontairement plafonnée à 35
\%, même si l'éclairage est allumé, afin de réduire la consommation
électrique.\hspace{0pt}

●\hspace{0pt} La prise du salon est systématiquement désactivée, ce qui
permet d'éliminer les consommations inutiles liées aux appareils en
veille.

Ce mode montre la capacité du système à adapter le comportement des
actionneurs en fonction de règles prédéfinies, avec un objectif clair
d'optimisation. Il illustre également l'intérêt de l'automatisation dans
une maison intelligente, en réduisant les interventions manuelles de
l'utilisateur.

Exemples :

\includegraphics[width=6.59444in,height=2.70833in]{vertopal_e39372513fc74674888784101d54e19c/media/image11.png}

\includegraphics[width=6.59444in,height=2.52778in]{vertopal_e39372513fc74674888784101d54e19c/media/image12.png}

Le mode Alarme est un mode de sécurité principalement lié à la détection
de forte présence de monoxyde de carbone (CO). Contrairement aux autres
modes, il ne peut pas être activé manuellement par l'utilisateur : il
est déclenché automatiquement par le système lorsqu\textquotesingle un
seuil critique est dépassé (c'est pour cela que le mode simulation
existe).

Lorsque le niveau de monoxyde de carbone atteint ou dépasse 200 ppm, le
mode Alarme s'active automatiquement et entraîne plusieurs actions
immédiates :

●\hspace{0pt} Activation de l'alarme de sécurité.\hspace{0pt}

●\hspace{0pt} Arrêt automatique du chauffage, de la climatisation et de
certaines prises afin de limiter les sources potentielles de
danger.\hspace{0pt}

●\hspace{0pt} Ouverture maximale des volets pour favoriser
l'aération.\hspace{0pt}

●\hspace{0pt} Activation de l'éclairage à pleine intensité pour signaler
visuellement la situation.\hspace{0pt}

●\hspace{0pt} Génération d'alertes visibles dans l'interface,
accompagnées de messages de prévention incitant à évacuer les lieux et à
contacter les secours.

Le mode Alarme se désactive automatiquement lorsque le niveau de
monoxyde de carbone redescend sous le seuil critique, ce qui évite de
maintenir inutilement une alerte. Ce\\
fonctionnement met en évidence la capacité du système à gérer des
situations critiques et aider à la prise de décision, renforçant ainsi
la sécurité globale de l'habitation.

Nous avons donc ajouté un mode simulation qui permet de montrer ce
qu\textquotesingle il se passe lorsque le taux de monoxyde de carbone
est trop élevé (voir dans la partie exemples
d\textquotesingle exécutions)

Enfin, sans avoir besoin d'activer aucun mode, le programme répond
d'emblée à certaines règles qu\textquotesingle on a ajouté dans le code:

``si luminosité \textless{} 40 → allumer lumière''\hspace{0pt}

``si présence == false → couper prise''\hspace{0pt}

``si température \textless{} 18 → activer chauffage''

\textbf{La répartition du travail entre les membres}

Lina : implémentation de l'IA avec ollama, du détecteur de CO, rédaction
du rapport, version finale du projet, développement de l'algorithme pour
le mode éco ,pour le mode simulation,et

\includegraphics[width=0.17639in,height=0.125in]{vertopal_e39372513fc74674888784101d54e19c/media/image6.png}\includegraphics[width=1.52083in,height=0.26042in]{vertopal_e39372513fc74674888784101d54e19c/media/image8.png}

pour l'instauration des règles générales , tests et validation de
l'affichage de données, ajout de la commande vocale, gestions des
alertes et messages de prévention et integrztion de la\\
commande vocale.

Sanabel : création de la base du projet/back-end, rédaction du rapport
et conversion en LaTeX, développement de l'interface web en HTML,
gestion de la communication entre le backend et l'interface, définition
de l'architecture globale du projet, intégration des capteurs et
actionneurs dans l'architecture

\textbf{6\hspace{0pt}Difficultés rencontrées}

Pendant le développement du projet, nous avons rencontré plusieurs
difficultés. Tout d'abord, l'ajout de nouvelles fonctions posait souvent
problème, notamment à cause des erreurs d'indentation. Ces erreurs
empêchaient parfois le programme de fonctionner correctement et
nécessitaient un temps de débogage important.

De plus, une première approche consistant à ajouter le code HTML dans un
fichier séparé a échoué en raison de problèmes de communication avec le
serveur Python.

Pour résoudre ce problème, nous avons opté pour l'intégration du code
HTML directement dans le fichier principal (main.py), ce qui a rétabli
la liaison et a permis le bon fonctionnement de l'interface.

Ensuite, certaines modifications apportées au code n'apparaissaient pas
toujours sur l'interface. Même après avoir changé le code, le résultat
n'était pas visible\\
immédiatement, compliquant le débogage et rendant le travail plus lent.

Nous avons également eu des difficultés avec la gestion de la voix. Il
arrivait que la voix se répète plusieurs fois ou qu'elle continue de
parler après avoir cliqué sur « fin », notamment après avoir mis
``\emph{off''} au bouton activer mode simulation.

Concernant l'intelligence artificielle implémentée avec Ollama, elle
fonctionne mais ne répond pas à toutes les questions qu'on lui pose.
Nous ne sommes pas parvenues à identifier la source du problème.

Enfin, la plus grande difficulté a été d'ajouter des fonctionnalités
plus avancées. À chaque nouvelle amélioration, il fallait souvent
modifier une grande partie du code existant,bien que le code était codé
par les ia tel que chatgpt et deep seek on rencontrait toujours des
incohérence ce qui entraînait plusieurs modifiaction . Cela rendait le
projet plus complexe à faire évoluer et demandait beaucoup d'ajustements
avant d'obtenir un résultat stable.
\end{quote}

\begin{longtable}[]{@{}
  >{\raggedright\arraybackslash}p{(\columnwidth - 2\tabcolsep) * \real{0.5000}}
  >{\raggedright\arraybackslash}p{(\columnwidth - 2\tabcolsep) * \real{0.5000}}@{}}
\toprule()
\begin{minipage}[b]{\linewidth}\raggedright
\textbf{7\hspace{0pt}}
\end{minipage} & \begin{minipage}[b]{\linewidth}\raggedright
\begin{quote}
\textbf{Bilan}
\end{quote}
\end{minipage} \\
\midrule()
\endhead
\bottomrule()
\end{longtable}

\begin{quote}
L'objectif principal de ce projet était de créer une maison intelligente
capable de réagir\\
automatiquement à différents paramètres (température, présence, qualité
de l'air, etc.) tout en intégrant une interaction avec une intelligence
artificielle. Ce projet nous a confrontées à la réalité d'un système
complet, allant du code backend jusqu'à l'interface utilisateur, ainsi
qu'à la logique derrière l'automatisation et la gestion des capteurs et
actionneurs.

Grâce à ce travail, nous avons appris à utiliser efficacement des outils
d'IA générative pour

\includegraphics[width=0.17639in,height=0.125in]{vertopal_e39372513fc74674888784101d54e19c/media/image6.png}\includegraphics[width=1.52083in,height=0.26042in]{vertopal_e39372513fc74674888784101d54e19c/media/image8.png}

produire du code, à formuler des prompts précis pour obtenir des
résultats exploitables et à tester, corriger et adapter le code généré à
nos besoins. Nous avons aussi pris conscience de
l\textquotesingle importance de bien organiser un projet pour faciliter
son évolution.

Pour la suite, l'idée serait d'avoir des conversations plus fluides avec
l'assistant, avec davantage de connaissances intégrées et pas seulement
celles disponibles dans l'immédiat. À terme, l'assistant répondrait non
seulement à la voix, mais également à des requêtes écrites, et pas
uniquement via des boutons à activer ou désactiver. Cela rendrait
l'interaction plus naturelle et plus proche d'un véritable assistant
intelligent.

Enfin, l'utilisation d'Ollama est très importante dans ce projet. Cet
outil permet d'intégrer une intelligence artificielle locale et ouvre
beaucoup de possibilités d'évolution. Avec Ollama, il serait possible
d'enrichir les réponses, d'améliorer la compréhension des demandes de
l'utilisateur et d'ajouter des fonctionnalités plus avancées. À long
terme, Ollama pourrait devenir le cœur du système intelligent et
permettre un développement encore plus poussé et plus interactif de la
maison intelligente.
\end{quote}

\begin{longtable}[]{@{}
  >{\raggedright\arraybackslash}p{(\columnwidth - 2\tabcolsep) * \real{0.5000}}
  >{\raggedright\arraybackslash}p{(\columnwidth - 2\tabcolsep) * \real{0.5000}}@{}}
\toprule()
\begin{minipage}[b]{\linewidth}\raggedright
\includegraphics[width=0.17639in,height=0.28194in]{vertopal_e39372513fc74674888784101d54e19c/media/image13.png}
\end{minipage} & \begin{minipage}[b]{\linewidth}\raggedright
\begin{quote}
\includegraphics[width=1.60417in,height=0.28194in]{vertopal_e39372513fc74674888784101d54e19c/media/image14.png}
\end{quote}
\end{minipage} \\
\midrule()
\endhead
\bottomrule()
\end{longtable}

\begin{longtable}[]{@{}
  >{\raggedright\arraybackslash}p{(\columnwidth - 2\tabcolsep) * \real{0.5000}}
  >{\raggedright\arraybackslash}p{(\columnwidth - 2\tabcolsep) * \real{0.5000}}@{}}
\toprule()
\begin{minipage}[b]{\linewidth}\raggedright
\includegraphics[width=0.17639in,height=0.28194in]{vertopal_e39372513fc74674888784101d54e19c/media/image15.png}
\end{minipage} & \begin{minipage}[b]{\linewidth}\raggedright
\begin{quote}
\includegraphics[width=1.58333in,height=0.28194in]{vertopal_e39372513fc74674888784101d54e19c/media/image16.png}
\end{quote}
\end{minipage} \\
\midrule()
\endhead
\bottomrule()
\end{longtable}

\begin{quote}
{[}CAT{]} savoircoder.fr/cat

\includegraphics[width=0.3125in,height=0.21944in]{vertopal_e39372513fc74674888784101d54e19c/media/image17.png}

\textbf{Annexe A :\hspace{0pt}} \textbf{Cahier des charges}\\
\textbf{1. Contexte du projet}\\
Ce projet a pour but de réaliser une maison intelligente simulée. Le
système permet de surveiller différents paramètres de l'habitat et
d'agir automatiquement en fonction de règles définies.

Une interface web permet à l'utilisateur de contrôler la maison, et une
intelligence artificielle assiste l'utilisateur dans ses demandes.

\textbf{2. Objectifs du projet}\\
●\hspace{0pt} Créer une simulation réaliste d'une maison
intelligente\hspace{0pt}●\hspace{0pt} Gérer automatiquement des capteurs
et des appareils\hspace{0pt}●\hspace{0pt} Améliorer la sécurité de
l'habitat (alertes, alarmes)\hspace{0pt}●\hspace{0pt} Proposer une
interface simple et compréhensible\hspace{0pt}●\hspace{0pt} Intégrer une
intelligence artificielle locale\hspace{0pt}

\textbf{3. Besoins fonctionnels}\\
Le système doit permettre :\\
●\hspace{0pt} L'affichage des capteurs en temps réel (température,
luminosité, humidité, qualité de l'air, CO, etc.)\hspace{0pt}\\
●\hspace{0pt} Le contrôle des appareils (éclairage, chauffage,
climatisation, volets, prises, sécurité)\hspace{0pt}\\
●\hspace{0pt} La gestion de différents modes de fonctionnement
(automatique, éco, vacances, alarme)\hspace{0pt}\\
●\hspace{0pt} Le déclenchement d'alertes visuelles et vocales en cas de
danger\hspace{0pt}\\
●\hspace{0pt} L'interaction avec une intelligence artificielle (réponses
et actions)\\
\textbf{4. Besoins non fonctionnels}\\
●\hspace{0pt} Le système doit être fiable et stable\hspace{0pt}\\
●\hspace{0pt} L'interface doit être simple d'utilisation\hspace{0pt}\\
●\hspace{0pt} Les réponses doivent être rapides\hspace{0pt}\\
●\hspace{0pt} Le projet doit être évolutif et modifiable
facilement\hspace{0pt}

\includegraphics[width=0.3125in,height=0.21944in]{vertopal_e39372513fc74674888784101d54e19c/media/image17.png}

●\hspace{0pt} L'IA doit fonctionner en local, sans dépendre
d'Internet\hspace{0pt}

\textbf{5. Contraintes techniques}\\
●\hspace{0pt} Utilisation du langage Python\hspace{0pt}\\
●\hspace{0pt} Interface web en HTML, CSS et JavaScript\hspace{0pt}\\
●\hspace{0pt} Serveur web local (localhost)\hspace{0pt}\\
●\hspace{0pt} Utilisation d'Ollama pour l'intelligence
artificielle\hspace{0pt}\\
●\hspace{0pt} Fonctionnement sur un ordinateur standard\hspace{0pt}

\textbf{6. Sécurité}\\
●\hspace{0pt} Détection du monoxyde de carbone\hspace{0pt}\\
●\hspace{0pt} Déclenchement automatique d'une alarme au-delà d'un seuil
critique\hspace{0pt}●\hspace{0pt} Coupure des appareils
dangereux\hspace{0pt}\\
●\hspace{0pt} Alerte vocale et visuelle pour prévenir
l'utilisateur\hspace{0pt}

\textbf{7. Évolutions possibles}\\
●\hspace{0pt} Enrichissement des connaissances de
l'assistant\hspace{0pt}●\hspace{0pt} Amélioration des échanges (plus
naturels et fluides)\hspace{0pt}●\hspace{0pt} Ajout de commandes écrites
et vocales complètes\hspace{0pt}●\hspace{0pt} Amélioration de la
structure du code\hspace{0pt}\\
●\hspace{0pt} Développement de nouvelles fonctionnalités\\
\textbf{Annexe B :\hspace{0pt}Exemple d\textquotesingle exécution}

Lancement du serveur Python puis accès à l'interface web via un
navigateur: lorsque le programme est lancé, un message vocal est
entendu: ``Bonjour, la simulation de la maison intelligente a démarré.

\includegraphics[width=0.3125in,height=0.21944in]{vertopal_e39372513fc74674888784101d54e19c/media/image17.png}

\includegraphics[width=6.59444in,height=1.70833in]{vertopal_e39372513fc74674888784101d54e19c/media/image18.png}

À ce stade, aucun mode spécifique n'est activé : il s'agit de
l'interface principale par défaut . Depuis celle-ci, il est possible
d'ajuster différents paramètres, tels que la luminosité, la
climatisation, etc.

\includegraphics[width=6.59444in,height=2.625in]{vertopal_e39372513fc74674888784101d54e19c/media/image19.png}

\includegraphics[width=6.59444in,height=2.40694in]{vertopal_e39372513fc74674888784101d54e19c/media/image20.png}

Voici ce qu'il se passe lorsque le mode simulation est activé (ce mode
permet de montrer ce qui se produit en cas de présence excessive de
monoxyde de carbone).

Tout d'abord, un message vocal se déclenche : il indique la quantité de
monoxyde de carbone détectée et

\includegraphics[width=0.3125in,height=0.21944in]{vertopal_e39372513fc74674888784101d54e19c/media/image17.png}

informe l'utilisateur du danger. Ce message est également affiché à
l'écran.

\includegraphics[width=6.59444in,height=0.75972in]{vertopal_e39372513fc74674888784101d54e19c/media/image21.png}

exemple d'utilisation d'ollama dans l'application:

\includegraphics[width=6.59444in,height=2.07222in]{vertopal_e39372513fc74674888784101d54e19c/media/image22.png}

\includegraphics[width=6.59444in,height=1.19722in]{vertopal_e39372513fc74674888784101d54e19c/media/image23.png}

\textbf{projet Annexe C :\hspace{0pt} Manuel utilisateur}

\emph{\uline{1.Lancement du projet}}

1.\hspace{0pt} Lancer le programme Python

2.\hspace{0pt} Le serveur démarre automatiquement

3.\hspace{0pt} L'interface web s'ouvre dans le navigateur à l'adresse
:\hspace{0pt}

\emph{\uline{2. Interface principale}}

L'interface est divisée en plusieurs parties :

\includegraphics[width=0.3125in,height=0.21944in]{vertopal_e39372513fc74674888784101d54e19c/media/image17.png}

\textbf{Capteurs} : affichent les valeurs actuelles (température,
humidité, CO, etc.)

\textbf{Appareils} : permettent d'activer ou désactiver les équipements

●\hspace{0pt} \textbf{Modes} : permettent de changer le comportement
global de la maison ●\hspace{0pt} \textbf{Alertes} : affichent les
messages importants (ex : danger CO)

\textbf{3. Utilisation des capteurs}

●\hspace{0pt} Certains capteurs peuvent être modifiés manuellement
(température, luminosité, présence)\\
●\hspace{0pt} D'autres sont uniquement en lecture (humidité, qualité de
l'air)

\textbf{4. Utilisation des appareils}

●\hspace{0pt} Cliquer sur les boutons pour allumer ou éteindre les
appareils\\
●\hspace{0pt} Le volet peut être réglé avec un curseur\\
●\hspace{0pt} Les appareils réagissent aussi automatiquement selon les
modes actifs

\uline{5. Modes disponibles}

●\hspace{0pt}\textbf{\uline{Mode automatique}} : la maison s'adapte
seule\\
●\hspace{0pt}\uline{\textbf{Mode éco}} : réduit la consommation
d'énergie\\
●\hspace{0pt}\textbf{\uline{Mode vacances}} : sécurise la maison et
coupe les appareils ●\hspace{0pt}\textbf{\uline{Mode alarme}} : activé
automatiquement en cas de danger ●\hspace{0pt}\uline{\textbf{Simulation
CO} :} permet de tester une situation dangereuse
\end{quote}

\end{document}
